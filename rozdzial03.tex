\chapter{Założenia projektowe}
\label{chap:zalozenia-projektowe}
\section{Zarys architektury systemu}
\label{sec:zarys-arch}
Tworzony system (tj.\ aplikacja budżetu domowego) będzie składać się z dwóch niezależnych części: aplikacji klienckiej (dostarczającej interfejs graficzny dla użytkownika w przeglądarce) oraz API (zajmującego się przetwarzaniem logiki biznesowej i dostarczaniem potrzebnych danych). Zarys jego architektury przedstawiono na rysunku~\ref{fig:ogolna-arch}.
\begin{figure}[h]
	\centering
	\includegraphics[width=.9\linewidth]{rys03/ogolna-arch.pdf}
	\caption{Zarys architektury systemu}
	\label{fig:ogolna-arch}
\end{figure}

Użytkownik systemu będzie pracował w przeglądarce internetowej swojego komputera lub~urządzenia mobilnego. W trakcie jego pracy przeglądarka wysyłać będzie żądania do aplikacji klienckiej w celu pobrania potrzebnych danych i widoków, a także wysłania poleceń z~pomocą żądań HTTP POST.

Aplikacja kliencka przyjmować będzie żądania HTTP pochodzące z przeglądarki użytkownika. Następnie będzie komunikować się z API, w celu pobrania potrzebnych danych lub wykonania operacji, której żąda użytkownik. Po odpowiedzi ze strony API, aplikacja kliencka będzie zwracać gotowy widok HTML napełniony niezbędnymi dla użytkownika danymi. Ponadto Aplikacja kliencka będzie działać po stronie serwera, ponieważ tylko taka aplikacja może być zaufanym klientem z punktu widzenia protokołu OAuth 2.0. Wybór takiej opcji zapewni potencjalnie użycie najbezpieczniejszego typu autoryzacji i uwierzytelniania OpenId Connect, czyli \texttt{Hybrid Flow}. 

API, po przyjęciu żądań HTTP, będzie komunikować się przez warstwę modelu z bazą danych. Wykonywane będą operacje aktualizujące bazę danych, jak i tylko te pobierające z niej informacje. Następnie API będzie zwracać odpowiednią odpowiedź klientowi. Dane zwracane będą w ciele odpowiedzi HTTP w formacie JSON.

Baza danych będzie dostępna tylko przez jej własne API. Nie będzie widoczna z zewnątrz, co potencjalnie ma zwiększyć bezpieczeństwo przechowywanych w niej danych.

\section{Wymagania funkcjonalne}
\label{sec:wymagania-funkcjonalne}
Poniżej opisano wymagania funkcjonalne, jakie powinien spełniać tworzony system. Wykorzystano przy tym diagramy przypadków użycia. Zaprezentowane przypadki obejmują funkcje czterech wyróżnionych podsystemów: zarządzania kontem pieniężnym, zarządzania środkami finansowymi, zarządzania rodziną i zarządzania kategoriami. Przypadki uszczegółowiono posługując się schematem: nazwa przypadku użycia, aktor wykonujący akcję, warunki początkowe, warunki końcowe, cel oraz przebieg. 

\subsection{Wymagania dotyczące konta pieniężnego}
\label{subsec:wymagania-konto}
Poniżej opisano przypadki użycia dotyczące zarządzania kontem pieniężnym użytkownika. Zaprezentowano je także na diagramie (rys.~\ref{fig:use-case-account}).
\begin{enumerate}[labelwidth=1em,label=\arabic*.]
\item \textbf{Nazwa:} Dodaj konto. \newline
    \textbf{Aktor:} Zalogowany użytkownik. \newline
    \textbf{Warunki początkowe:} Użytkownik musi się wcześniej zarejestrować i być zalogowany w~momencie tworzenia konta. \newline
    \textbf{Warunki końcowe:} Stan konta musi być zgodny z polityką dla jego typu. \newline
    \textbf{Cel:} Stworzenie w systemie konta pieniężnego przypisanego do aktora wykonującego tę~operację, aby umożliwić zapisywanie informacji o przepływie gotówki. \newline
    \textbf{Przebieg:} Użytkownik wypełnia formularz. Wpisuje wymagane pola: nazwa konta, saldo początkowe i limit debetu oraz opcjonalne pola: numer konta, waluta i opis. Wybiera także typ konta.
\item \textbf{Nazwa:} Usuń konto. \newline
    \textbf{Aktor:} Właściciel konta. \newline
    \textbf{Warunki początkowe:} Użytkownik musi posiadać konto pieniężne. \newline
    \textbf{Warunki końcowe:} Użytkownik musi potwierdzić usunięcie konta. \newline
    \textbf{Cel:} Usunięcie konta pieniężnego użytkownika wraz z jego historią. \newline
    \textbf{Przebieg:} Użytkownik wybiera jedno ze swoich kont, a następnie naciska guzik usuń. Użytkownik musi potwierdzić tę operację.
\item \textbf{Nazwa:} Edytuj dane konta. \newline
    \textbf{Aktor:} Właściciel konta. \newline
    \textbf{Warunki początkowe:} Użytkownik musi posiadać konto pieniężne. \newline
    \textbf{Warunki końcowe:} Stan konta nie może przekraczać nowego limitu debetu. \newline
    \textbf{Cel:} Aktualizacja danych konta, bez utraty jego historii. \newline
    \textbf{Przebieg:} Użytkownik edytuje pola obowiązkowe formularza: nazwa konta, limit debetu oraz opcjonalne pola: numer konta, waluta i opis. Nie może edytować jego typu.
\item \label{last-item1}\textbf{Nazwa:} Zmień typ konta. \newline
    \textbf{Aktor:} Właściciel konta. \newline
    \textbf{Warunki początkowe:} Użytkownik musi posiadać konto pieniężne. \newline
    \textbf{Warunki końcowe:} Typ konta musi pozwalać na potencjalnie istniejący na koncie debet. \newline
    \textbf{Cel:} Zmiana typu konta na inny. Umożliwiające posiadanie debetu, lub oprocentowane. \newline
    \textbf{Przebieg:} Użytkownik po wybraniu konta zmienia jego typ, a swój wybór zatwierdza naciśnięciem guzika ,,Akceptuj''. 
\end{enumerate}

\begin{figure}[t]
	\centering
	\includegraphics[width=.65\linewidth]{rys03/use-case-account-2.pdf}
	\caption{Diagram przypadków użycia konta pieniężnego}
	\label{fig:use-case-account}
\end{figure}

\subsection{Wymagania dotyczące środków finansowych}
\label{subsec:wymagania-srodki-finansowe}
Na rysunku~\ref{fig:use-case-money} przedstawiono przypadki użycia związane z definiowaniem przepływu gotówki dla kont pieniężnych. Poniżej znajdują się także ich opisy.

\begin{enumerate}[labelwidth=1em,label=\arabic*.]
\item \textbf{Nazwa:} Dodaj wydatek/przychód. \newline
    \textbf{Aktor:} Uprawniony przez właściciela członek rodziny. \newline
    \textbf{Warunki początkowe:} Administrator rodziny zdefiniował co najmniej jedną kategorię dla~operacji finansowych. \newline
    \textbf{Warunki końcowe:} Stan konta po operacji dodania wydatku musi być zgodny z polityką dla typu konta. \newline
    \textbf{Cel:} Zdefiniowanie operacji pieniężnej w ramach konta. \newline
    \textbf{Przebieg:} Użytkownik tworzy wydatek lub przychód wpisując obowiązkowe pola: kwota i~data, a także te dodatkowe: notatka, odbiorca. Użytkownik musi także wybrać kategorię. 
\item \textbf{Nazwa:} Usuń wydatek/przychód. \newline
    \textbf{Aktor:} Uprawniony przez właściciela członek rodziny. \newline
    \textbf{Warunki początkowe:} Konto posiada co najmniej jeden wpis: wydatek lub przychód. \newline
    \textbf{Warunki końcowe:} Stan konta po operacji usunięcia przychodu musi być zgodny z polityką dla typu konta. \newline
    \textbf{Cel:} Usunięcie operacji pieniężnej w ramach konta. \newline
    \textbf{Przebieg:} Użytkownik wybiera wydatek lub przychód, a następnie naciska guzik usuń. 
\item \textbf{Nazwa:} Edytuj wydatek/przychód. \newline
    \textbf{Aktor:} Uprawniony przez właściciela członek rodziny. \newline
    \textbf{Warunki początkowe:} Konto posiada co najmniej jeden wpis: wydatek lub przychód. \newline
    \textbf{Warunki końcowe:} Stan konta po edycji wpisu musi być zgodny z polityką dla typu konta. \newline
    \textbf{Cel:} Edycja operacji pieniężnej w ramach konta. \newline
    \textbf{Przebieg:} Użytkownik wybiera wydatek lub przychód, a następnie naciska guzik ,,edytuj''. Edytuje pola obowiązkowe i dodatkowe. Jeśli chce zmienić kategorię musi ją wybrać z listy. \newline
    \textbf{Rozszerzenia: } 
    \begin{enumerate}[label=\alph*)]
        \item Użytkownik edytuje kategorię wpisu: Wybór kategorii.
    \end{enumerate}
\item \textbf{Nazwa:} Wykonaj przelew. \newline
    \textbf{Aktor:} Uprawniony przez właściciela członek rodziny. \newline
    \textbf{Warunki początkowe:} Administrator rodziny zdefiniował co najmniej jedną kategorię dla operacji finansowych. \newline
    \textbf{Warunki końcowe:} Stan konta po przelewie musi być zgodny z polityką dla typu konta. \newline
    \textbf{Cel:} Wykonanie przelewu między kontami lub zewnętrznego. \newline
    \textbf{Przebieg:} Użytkownik definiuje przelew wpisując obowiązkowe pola: kwota i data, a także dodatkowe: notatka, odbiorca. Użytkownik musi także wybrać kategorię. \newline
    \textbf{Rozszerzenia: }
    \begin{enumerate}[label=\alph*)]
        \item Użytkownik definiuje przelew między kontami rodziny: Przelew własny.
    \end{enumerate}
\end{enumerate}

\begin{figure}[t]
	\centering
	\includegraphics[width=.7\linewidth]{rys03/use-case-money-2.pdf}
	\caption{Subdiagram przypadków użycia dla operacji finansowych}
	\label{fig:use-case-money}
\end{figure}

\subsection{Wymagania dotyczące rodziny}
\label{subsec:wymagania-rodzina}
Wymagania funkcjonalne dotyczące obsługi w aplikacji grupy rodziny, dalej zwanej rodziną, przedstawione zostały na~ rysunku~\ref{fig:use-case-family}. Poniżej znajdują się także ich szczegółowe opisy.

\begin{enumerate}[labelwidth=1em,label=\arabic*.]
\item \textbf{Nazwa:} Stwórz rodzinę. \newline
    \textbf{Aktor:} Zalogowany użytkownik. \newline
    \textbf{Warunki początkowe:} Użytkownik zarejestrował się w aplikacji. \newline
    \textbf{Warunki końcowe:} Użytkownik staje się administratorem nowoutworzonej rodziny. \newline
    \textbf{Cel:} Stworzenie rodziny, w ramach której można współdzielić konta pieniężne i ich historie. \newline
    \textbf{Przebieg:} Rodzina zostaje stworzona w momencie rejestracji. 
\item \textbf{Nazwa:} Akceptuj zaproszenie. \newline
    \textbf{Aktor:} Zalogowany użytkownik. \newline
    \textbf{Warunki początkowe:} Administrator innej rodziny dodał do niej użytkownika. \newline
    \textbf{Warunki końcowe:} Użytkownik zaakceptował zaproszenie.  \newline
    \textbf{Cel:} Dołączenie do nowej rodziny. \newline
    \textbf{Przebieg:} Użytkownik wyświetla zaproszenia do rodziny. Następnie akceptuje zaproszenie do jednej, bądź wielu z nich.
\item \textbf{Nazwa:} Dodaj członka. \newline
    \textbf{Aktor:} Administrator rodziny. \newline
    \textbf{Warunki początkowe:} Użytkownik stworzył rodzinę. \newline
    \textbf{Warunki końcowe:} Użytkownik akceptuje zaproszenie do rodziny.  \newline
    \textbf{Cel:} Dodanie nowego członka do rodziny, aby współdzielić historię finansową. \newline
    \textbf{Przebieg:} Administrator wpisuję login użytkownika, którego chce dodać i zatwierdza wybór guzikiem. 
\item \textbf{Nazwa:} Usuń członka. \newline
    \textbf{Aktor:} Administrator rodziny. \newline
    \textbf{Warunki początkowe:} Użytkownik dodał do rodziny co~najmniej jednego innego członka. \newline
    \textbf{Warunki końcowe:} --  \newline
    \textbf{Cel:} Usunięcie członka z rodziny. \newline
    \textbf{Przebieg:} Administrator naciska guzik usuń członka, a następnie potwierdza swoją akcję.
\item \textbf{Nazwa:} Edytuj uprawnienia dostępu do konta. \newline
    \textbf{Aktor:} Członek rodziny będący właścicielem konta. \newline
    \textbf{Warunki początkowe:} Użytkownik jest właścicielem konta. W rodzinie, do której przypisane jest konto jest więcej niż jeden członek. \newline
    \textbf{Warunki końcowe:} Właściciel konta potwierdza nadanie uprawnień naciskając guzik. \newline
    \textbf{Cel:} Nadanie uprawnień członkowi rodziny, aby mógł publikować wpisy w ramach konta i/lub przeglądać jego historię.  \newline
    \textbf{Przebieg:} Właściciel konta wybiera z listy jedno ze swoich kont w ramach rodziny. Następnie z listy wybiera członka rodziny, któremu chce nadać uprawnienia. Nadaje członkowi uprawnienia spośród: prawo do wglądu, prawo do dodawania wpisów.
    Swój wybór zatwierdza guzikiem ,,zatwierdź''.
\end{enumerate}

\begin{figure}[t]
	\centering
	\includegraphics[width=.65\linewidth]{rys03/use-case-family-2.pdf}
	\caption{Diagram przypadków użycia dla rodziny}
	\label{fig:use-case-family}
\end{figure}

\subsection{Wymagania dotyczące kategorii}
\label{subsec:wymagania-kategorie}

\begin{enumerate}[labelwidth=1em,label=\arabic*.]
\item \textbf{Nazwa:} Dodaj kategorię. \newline
    \textbf{Aktor:} Zalogowany użytkownik. \newline
    \textbf{Warunki początkowe:} Użytkownik zarejestrował się w aplikacji. \newline
    \textbf{Warunki końcowe:} Nazwa kategorii w swoim poddrzewie jest unikatowa. \newline
    \textbf{Cel:} Stworzenie kategorii, aby łatwiej segregować wydatki i przychody. \newline
    \textbf{Przebieg:} Użytkownik naciska guzik dodaj kategorię na drzewie kategorii. Następnie wpisuje jej nazwę i wybiera kolor wyświetlania wpisów przypisanych do niej.
\item \textbf{Nazwa:} Edytuj kategorię. \newline
    \textbf{Aktor:} Zalogowany użytkownik. \newline
    \textbf{Warunki początkowe:} Użytkownik dodał co najmniej jedną kategorię. \newline
    \textbf{Warunki końcowe:} Nazwa kategorii w swoim poddrzewie jest unikatowa. \newline
    \textbf{Cel:} Edycja istniejącej kategorii bez utraty historii wpisów tej kategorii. \newline
    \textbf{Przebieg:} Użytkownik edytuje nazwę i kolor kategorii. Może także przeciągnąć ją do innej kategorii nadrzędnej lub zrobić z niej kategorię najwyższego rzędu.
\item \textbf{Nazwa:} Ukryj kategorię. \newline
    \textbf{Aktor:} Zalogowany użytkownik. \newline
    \textbf{Warunki początkowe:} Użytkownik dodał co najmniej jedną kategorię. \newline
    \textbf{Warunki końcowe:} -- \newline
    \textbf{Cel:} Ukrycie istniejącej kategorii bez utraty historii wpisów tej kategorii. \newline
    \textbf{Przebieg:} Użytkownik w menu kategorii klika ukryj.
\item \textbf{Nazwa:} Usuń kategorię. \newline
    \textbf{Aktor:} Zalogowany użytkownik. \newline
    \textbf{Warunki początkowe:} Użytkownik dodał co najmniej jedną kategorię. Kategoria nie posiada żadnych wpisów. \newline
    \textbf{Warunki końcowe:} Nie istnieje wpis, który nie ma przypisanej kategorii. \newline
    \textbf{Cel:} Usunięcie kategorii, do której nie jest przypisany żaden wpis. \newline
    \textbf{Przebieg:} Użytkownik w menu kategorii klika usuń i zatwierdza wybór.
\end{enumerate}

\begin{figure}[t]
	\centering
	\includegraphics[width=.65\linewidth]{rys03/use-case-category-2.pdf}
	\caption{Diagram przypadków użycia dla kategorii}
	\label{fig:use-case-category}
\end{figure}

\section{Wymagania niefunkcjonalne}
\label{sec:wymagania-niefunkcjonalne}

\begin{enumerate}[labelwidth=1em,label=\arabic*.]

\item System składać ma się z~dwóch głównych części:
\begin{enumerate}[label=\alph*)]
\item aplikacji serwerowej -- API. Jej zadaniem będzie dostarczanie interfejsu programistycznego do pobierania danych i~modyfikowania stanu systemu.
\item aplikacja webowej, dostarczająca graficzny interfejs użytkownika -- \texttt{GUI}. Aplikacja ta~będzie konsumować API.
\end{enumerate}

\item Zarówno aplikacja webowa, jak i API będą stworzone przy pomocy frameworka ASP.NET Core w wersji co najmniej 2.2: 
\begin{enumerate}[label=\alph*)]
\item Do napisania aplikacji webowej użyte zostaną \texttt{Razor Pages} oraz komponenty widoku (ang.~\emph{view components}). 
\item Warstwa persystencji danych w aplikacji API stworzona zostanie z~pomocą Entity Framework Core 2.2. Wykorzystane zostanie podejście \emph{code first} i mechanizm migracji.
\end{enumerate}

\item Pobieranie danych z API ma odbywać się za pomocą protokołu OData.

\item API używać będzie bazy danych na serwerze Microsoft SQL Server 2017 (lub nowszym).

\item Aplikacja webowa powinna poprawnie działać co najmniej w przeglądarkach: Mozilla Firefox, Google Chrome, Opera, Microsoft Edge oraz Safari, a~także przeglądarkach mobilnych: Opera Mobile, Google Chrome, Mozilla Firefox i Safari.

\item Aplikacja będzie wymuszać działanie z wykorzystaniem protokołu \texttt{HTTPS} (ang.~\emph{Hypertext Transfer Protocol Secure}).

\item Autoryzacja i uwierzytelnienie ma być przeprowadzona z wykorzystaniem protokołu OpenId Connect i frameworka autoryzacji OAuth 2.0. Użyty powinien zostać \emph{Hybrid Flow} lub~\emph{Authorization Code Flow}.

\item Dostawcą tożsamości w systemie powinna być osobna aplikacja, posiadająca własną bazę danych lub zewnętrzy system autoryzacji, np.~\texttt{Auth0}.

\item API ma posiadać dokumentację stworzoną przy pomocy narzędzia \texttt{Swagger}.

\item Kod źródłowy ma być napisany z wykorzystaniem dobrych praktyk programowania~i~wzorców projektowych ułatwiających jego utrzymanie i rozwijanie.

\item System działać będzie na platformie chmurowej \texttt{Microsfot Azure}.

\item Poszczególne składowe systemu w chmurze powinny być automatycznie skalowalne.

\item Aplikacja webowa powinna być bezpieczna i odporna na ataki. Zostanie wykorzystany mechanizm obrony przed \texttt{CSRF} (ang.~Cross-Site Request Forgery) -- \texttt{anti-forgery token}.

\end{enumerate}

