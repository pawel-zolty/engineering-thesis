\chapter{Implementacja}
\label{chap:implementacja}

\section{Struktura projektu}
\label{sec:struktura-projektu}
W tym podrozdziale przedstawiono logiczną i fizyczną strukturę projektu. Uwzględniono przy tym podział na projekty aplikacji klienckiej oraz API, gdyż stanowią one tak na prawdę niezależne rozwiązania (ang.~\emph{solution}) w środowisku programistycznym Visual Studio 2019.

\subsection{Struktura fizyczna projektu}
\label{sec:struktura-fizyczna-projektu}
Obie aplikacje (aplikacja kliencka i API) znajdują się w tym samym repozytorium Git. Pliki ich rozwiązań znajdują się w folderze głównym repozytorium. W folderze tym znajduję się także plik \texttt{.gitignore}. Projekty należące do rozwiązań znajdują się w folderze \texttt{src}. W katalogu \texttt{Zoltaniecki.IDP} znajduję się potrzebny do testów dostawca tożsamości, będący tylko wydmuszką prawdziwego serwisu. Strukturę głównego folderu można zobaczyć na rysunku~\ref{fig:fiz-1}.
\begin{figure}[ht]
	\centering
	\includegraphics[scale=.77]{rys04/struktura-fizyczna-1.PNG}
	\caption{Struktura fizyczna repozytorium}
	\label{fig:fiz-1}
\end{figure}

W folderze \texttt{src} znajdują się dwa następne foldery: \texttt{API} oraz \texttt{Client}. Znajdują się w nich kolejno projekty dla rozwiązania: \texttt{BudgetApplicationAPI.sln} i~dla rozwiązania \texttt{BudgetApplicationClient.sln} (rys.~\ref{fig:fiz-2}). 
%todo czy usunać?
\begin{figure}[ht]
	\centering
	\includegraphics[scale=.77]{rys04/struktura-fizyczna-2.PNG}
	\caption{Zawartość folderu src}
	\label{fig:fiz-2}
\end{figure}

Warto zwrócić uwagę, że zarówno w API jak i w aplikacji klienckiej klasy pogrupowane są według funckjonoalności, a nie typów np.: \texttt{Services}, \texttt{Controllers}, \texttt{Views}, \texttt{Models}. Daje to dużą elastyczność i wygodę przy pracy programisty, gdyż nie musi szukać klas związanych z~jedną funkcjonalnościach w wielu w poddrzewach katalogów.

\subsubsection{Fizyczna struktura API}
Aplikacja API składa się z czterech projektów (rys.~\ref{fig:fiz-api-2}): \texttt{BudgetApplication.API}, \texttt{BudgetApplication.Domain}, \texttt{BudgetApplication.Infrastructure} oraz \texttt{BudgetApplication.QueryInfrastructure} umieszczonych w osobnych folderach. % (rys.~\ref{fig:fiz-api-1})
%%%todo czy usunać?
%%\begin{figure}[ht]
	%%\centering
	%%\includegraphics[scale=.77]{rys04/struktura-fizyczna-api-1.PNG}
	%%\caption{Zawartość folderu API}
	%%\label{fig:fiz-api-1}
%%\end{figure}
%Na rysunku~\ref{fig:fiz-api-2} przedstawiono strukturę rozwiązania dla API. Widać na nim cztery projekty zawierające różne foldery. 
\begin{figure}[ht]
	\centering
	\includegraphics[scale=.77]{rys04/struktura-fizyczna-api-2.PNG}
	\caption{Struktura rozwiązania API}
	\label{fig:fiz-api-2}
\end{figure}

Foldery projektu \texttt{API} zawierają pogrupowane według funkcjonalności klasy kontrolerów API (ang.~\emph{controllers}), a także klasy związane z opisaną w sekcji~\ref{subsec:mediator} biblioteką \texttt{MediatR}: handlery (ang.~\emph{handlers}), komendy (ang.~\emph{commands}) i odpowiedzi(ang.~\emph{responses}). Katalog \texttt{Shared} zawiera współdzielone interfejsy. 

W projekcie \texttt{Domain} podkatalogi katalogu \texttt{AggregatesModel} zawierają pogrupowane według funkcjonalności encje, ich serwisy domenowe, klasy wyjątków domenowych, enumeratory, klasy typu \texttt{DTO} (ang.~\emph{Data transfer object}), a także inne klasy związane z przetwarzaniem logiki biznesowej. Folder \texttt{SeedWork} zawiera związane z taktycznym DDD interfejsy i klasy bazowe. \texttt{Program.cs} zawiera funkcję \texttt{Main}, czyli punkt wejścia do programu, a w klasie \texttt{Startup} konfigurowana jest aplikacja i jej serwisy.

Projekty \texttt{Infrastructure} i \texttt{QueryInfrastructure} zawierają warstwę dostępu do danych kolejno dla komend i zapytań, związanych z zastosowanym wzorcem \texttt{CQRS}, opisanym w punkcie~\ref{subsec:cqrs}. Foldery \texttt{EntityConfigurations} zawierają mapowania relacyjno-obiektowe charakterystyczne dla Entity Framework Core. Folder \texttt{Repositories} zawiera implementacje wzorca repozytorium dla klas encji biznesowych. Natomiast katalog \texttt{Migrations} zawiera pliki migracji modelu w bazie danych.


\subsubsection{Fizyczna struktura aplikacji klienckiej}

Aplikacja kliencka składa się z dwóch projektów (rys.~\ref{fig:fiz-client-2}): \texttt{BudgetApplication.Client} oraz \texttt{BudgetApplication.Client.Core} umieszczonych w osobnych folderach. 
%%\begin{figure}[ht]
	%%\centering
	%%\includegraphics[scale=.77]{rys04/struktura-fizyczna-client-1.PNG}
	%%\caption{Zawartość folderu Client}
	%%\label{fig:fiz-client-1}
%%\end{figure}
%Na rysunku~\ref{fig:fiz-client-2} przedstawiono strukturę rozwiązania aplikacji klienckiej.
\begin{figure}[htb]
	\centering
	\includegraphics[scale=.77]{rys04/struktura-fizyczna-client-2.PNG}
	\caption{Struktura rozwiązania aplikacji klienckiej}
	\label{fig:fiz-client-2}
\end{figure}

Pierwszy z projektów zawiera jeden główny folder z widokami i ich modelami strony (ang.~\emph{page model}), pogrupowanymi w foldery według funkcjonalności. W głównym folderze znajdują się też widoki niezwiązane z modelami strony, a także folder \texttt{Shared} z widokami częściowymi używanymi przez pozostałe widoki. Pliki: \texttt{Program.cs} i \texttt{Startup.cs} pełnią podobną funkcję co w projekcie API.

Drugi z projektów, także zawiera tylko jeden główny katalog. Zawiera on odpowiednio skonfigurowane klasy klientów HTTP, komunikujące się z API, a także klasy modeli, które są zwracane przez zakończone sukcesem żądania HTTP.

Klasa \texttt{ODataResult} jest klasą odpowiedzi żądań HTTP GET. Zawiera ona właściwości takie jak: \emph{Count}, \emph{Metadata}, \emph{Context} i \emph{Value}, ściśle związana z implementacją protokołu OData -- punkt~\ref{subsec:odata}. Pole Value jest polem generycznym w zależności od zwracanego przez żądanie HTTP modelu. 


\subsection{Struktura logiczna projektu}
\label{sec:struktura-logiczna-projektu}
Poniżej zaprezentowano logiczną strukturę projektu. Pokazano też zależności między projektami zawartymi w rozwiązaniach dla aplikacji klienckiej oraz API.

\subsubsection{Logiczna struktura API}
Aplikacja API składa się z czterech projektów: \texttt{BudgetApplication.API}, \texttt{BudgetApplication.Domain}, \texttt{BudgetApplication.Infrastructure} oraz \texttt{BudgetApplication.QueryInfrastructure} z zależnościami jak na rysunku~\ref{fig:api-arch}. 
\begin{figure}[ht]
	\centering
	\includegraphics[scale=.55]{rys04/api-arch.png}
	\caption{Zależności między projektami w API}
	\label{fig:api-arch}
\end{figure}

Projekt \texttt{API} zawiera API kontrolery, a także handlery, odpowiedzi i żądania MediatR. Razem składają się one na warstwę aplikacji. Warstwa aplikacji operuje na obiektach domenowych, dlatego też projekt \texttt{API} zależny jest od projektu \texttt{Domain}.

Projekt \texttt{Domain} jest projektem ślepym i nie ma żadnych zależności. Znajduje się w nim kod niezależny od innych zewnętrznych frameworków. Definiuje on zachowania biznesowe obiektów domenowych, które są używane podczas wykonywania komend CQRS, wcześniej opisanych w punkcie~\ref{subsec:cqrs}. Model domenowy jest szczególnie istotny z punktu widzenia zachowania prawidłowego stanu systemu w trakcie i po wykonaniu operacji.

Warstwę persystencji dla encji domenowych stanowi projekt \texttt{Infrastructure}. Znajdują się w nim m.in. repozytoria, które implementują kontrakt zdefiniowany w projekcie \texttt{Domain}. Repozytoria korzystają z mapera relacyjno-obiektowego Entity Framework Core. Ten projekt tworzony jest dla zachowania niezależności domeny od zewnętrznych frameworków.

Projekt \texttt{QueryInfrastructure} to projekt dostarczający uproszczoną wersję modelu zapisanego w bazie danych, bez logiki domenowej. Obiekty tworzone w tej części aplikacji potrzebne są dla wykonania zapytań CQRS -- punkt~\ref{subsec:cqrs}. Projekt stworzono, podobnie jak poprzedni, również z wykorzystaniem Entity Framework Core. Oba projekty są niezależne i nic nie stoi na~przeszkodzie, aby zamiast EF Core wykorzystać technologię wbudowaną w Net Framework -- ADO.NET czy bibliotekę innego mapera relacyjno-obiektowego -- \texttt{Dapper}.

\subsubsection{Logiczna struktura aplikacji klienckiej}

Aplikacja kliencka składa się z dwóch projektów: \texttt{BudgetApplication.Client} oraz \texttt{BudgetApplication.Client.Core} z zależnościami jak na rysunku~\ref{fig:client-arch}.

\begin{figure}[ht]
	\centering
	\includegraphics[scale=.55]{rys04/client-arch.png}
	\caption{Zależności między projektami w aplikacji klienckiej}
	\label{fig:client-arch}
\end{figure}

Projekt \texttt{Client} zawiera strony Razor Pages, które odpowiedzialne są za prezentowanie danych użytkownikowi i zapewnianie mu interfejsu do dokonywania zmian w systemie.

\texttt{Client.Core} zawiera klasy odpowiedzialne za komunikację z API i dostarczanie danych potrzebnych na stronach aplikacji.


\section{Szczegóły implementacji}
\label{sec:szczegoly-implementacji}
W tym podrozdziale opisano szczegóły implementacji API oraz aplikacji klienckiej. 

\subsection{Szczegóły implementacji API}
\label{subsec:szczegoly-implementacji-api}
Żądania HTTP wysłane do API przechodzą przez moduł routingu (ang.~\emph{Routing Middleware}) frameworka ASP.NET MVC. To on decyduje, do którego kontrolera i do jakiej jego akcji skierować żądanie. Jest to szczególnie istotne w bardziej skomplikowanych scenariuszach. Jednym z takich scenariuszy jest używanie implementacji protokołu OData równocześnie z routingiem atrybutowym (ang.~\emph{attribute routing}). Aby dodać routing OData do API, wystarczy w metodzie \texttt{Configure} klasy \texttt{Startup} dodać kod widoczny na listingu~\ref{list:odata-route-config}. Konfiguruje on prefix ścieżki do zasobów OData na \texttt{/api/odata}. Umożliwia on także na używanie selekcji, filtrowania, sortowania, zliczania, a także rozszerzania i stronicowania zasobu OData. 

Metoda \texttt{GetEdmModel} buduje model OData na podstawie klas encji przechowywanych w~bazie danych. Model mapowany jest do dalszej części ścieżki zasobów adresu URL za pomocą parametru przekazywanego do metody \texttt{EntitySet}. Wartość tego parametru musi być taka sama jak atrybut routingu w kontrolerze OData. Na podstawie modelu EDM generowane są metadane dla zasobów. 

{\belowcaptionskip=-10pt
\begin{lstlisting}[label=list:odata-route-config,
    caption=Konfiguracja rouingu OData w aplikacji MVC]
app.UseMvc(routeBuilder =>
{
    AddOdataRoutes(routeBuilder);
});
...
private static void AddOdataRoutes(Microsoft.AspNetCore.Routing.IRouteBuilder routeBuilder)
{
    routeBuilder.EnableDependencyInjection();
    routeBuilder.Select().Filter().OrderBy().Count().Expand().MaxTop(100).SkipToken();
    routeBuilder.MapODataServiceRoute("odata", "api/odata", OdataExtensionsMethods.GetEdmModel());
}
...
internal static IEdmModel GetEdmModel()
{
    var builder = new ODataConventionModelBuilder();

    builder.EntitySet<Category>("categories");
    builder.EntitySet<MoneyAccount>("money-accounts");
    builder.EntitySet<MoneyOperation>("money-operations");
    builder.EntitySet<Expense>("expenses");
    builder.EntitySet<Income>("incomes");

    return builder.GetEdmModel();
}
\end{lstlisting}
}

Kontroler OData i standardowy kontroler API różnią się od siebie pod kilkoma względami. Standardowy kontroler API dziedziczy po klasie \texttt{ControllerBase} i używa atrybutów routingu \texttt{Route} z przestrzeni nazw \texttt{Microsoft.AspNetCore.Mvc}. Natomiast kontroler OData dziedziczy po klasie \texttt{ODataController}. Dzięki zastosowaniu klasy \texttt{ODataController} kontroler OData m.in.~zwraca dane w~formacie zgodnym z protokołem OData, czyli opakowuje dane w~pole \texttt{values}, zapewnia link do metadanych zasobu, a także odpowiednie nagłówki charakterystyczne dla protokołu. Kontroler OData używa także innych atrybutów routingu: \texttt{ODataRoute} dla pojedynczej akcji oraz \texttt{ODataRoutePrefix} dla określenia prefixu routingu dla wszystkich akcji kontrolera. Znajdują się one w przestrzeni nazw \texttt{Microsoft.AspNet.OData.Routing}. Zastosowanie nieodpowiednich atrybutów routingu skutkuje błędem dopasowania wielu akcji do jednego adresu (ang.~\emph{AmbiguousActionException: Multiple actions matched}) lub zwróceniem statusu HTTP 404. Na listingu~\ref{list:api-ctrl-1} oraz~\ref{list:odata-ctrl-1} przedstawiono przykłady implementacji obu kontrolerów. Część kodu została pominięta dla klarowności.

Ponadto w kontrolerze OData używane są atrybuty \texttt{EnableQuery}. Zezwalają one na korzystanie w adresie URL z opcji filtrowania, sortowania itd., skonfigurowanych w metodzie \texttt{AddOdataRoutes} na listingu~\ref{list:odata-route-config}. Brak tych atrybutów dla konkretnej akcji skutkować będzie brakiem obsługi tych funkcjonalności. Ścieżka wyszukiwania (ang.~\emph{query string} będzie w tym wypadku ignorowana. Istnieje też atrybut \texttt{Queryable}, który można wykorzystać do ograniczenia funkcjonalności, które programista chce udostępnić w akcji. Robi się to za pomocą przekazania do atrybutu podzbioru skonfigurowanych opcji wyszukiwania OData. Tutaj przykład atrybutu ograniczającego opcje wyszukiwania do stronicowania: \texttt{[Queryable(AllowedQueryOptions=
    AllowedQueryOptions.Skip | AllowedQueryOptions.Top)]}

{\belowcaptionskip=-10pt
\begin{lstlisting}[label=list:api-ctrl-1,
    caption=Przykład implementacji standardowego kontrolera API]
[Route("api/money-accounts")]
[ApiController]
public class MoneyAccountsController : ControllerBase
{
    private readonly IMediator _mediator;
...
    //Patch api/money-accounts/1
    [HttpPatch]
    [Route("{moneyAccountId}")]
    public async Task<IActionResult> UpdateAccount(Guid moneyAccountId,
        UpdateMoneyAccountCommand updateMoneyAccountCommand)
    {
        updateMoneyAccountCommand.MoneyAccountId = moneyAccountId;
        bool result = await _mediator.Send(updateMoneyAccountCommand);
    
        if (!result)
        {
            return BadRequest();
        }
    
        return Ok();
    }
}
\end{lstlisting}
}

{\belowcaptionskip=-10pt
\begin{lstlisting}[label=list:odata-ctrl-1,
    caption=Przykład implementacji kontrolera OData]
[ODataRoutePrefix("money-accounts")]
public class ODataMoneyAccountController : ODataController
{
    private readonly QueryEFContext _moneyAccountRepository;
...
    [EnableQuery]
    [ODataRoute("{id}")]
    public ActionResult<MoneyAccount> GetAsync(Guid id)
    {
        return Ok(_moneyAccountRepository.MoneyAccounts
            .Where(acc => acc.Id == id));
    }
}
\end{lstlisting}
}

W systemie aplikacji domowego budżetu kontrolery OData są używane tylko do żądań HTTP GET, a kontrolery API do wszystkich innych żądań, które zmieniają stan systemu.
Kontroler OData bezpośrednio korzysta z kontekstu do bazy danych, który zwraca kolekcję implementują generyczny interfejs \texttt{IQueryable}. Dzięki temu OData jest w stanie przenieść logikę zapytania do bazy danych.

Standardowy kontroler API korzysta z mediatora, aby wysłać komendę do kolejnej warstwy aplikacji, w której zachodzą procesy biznesowe. Przykład serwisu aplikacji, który przetwarza komendę pokazano na listingu~\ref{list:handler-impl}. Operuje on na encjach i serwisach domenowych. W tym wypadku wykorzystano repozytorium do pobrania konta bankowego z bazy danych. Następnie do konta bankowego dodawany jest nowy wydatek, a całość zapisywana jest z zastosowaniem wzorca \texttt{Unit of Work}, który gwarantuje, że zmiany, które zaszły w systemie będą zapisane w~jednej transakcji bazodanowej. 

{\belowcaptionskip=-10pt
\begin{lstlisting}[label=list:handler-impl,
    caption=Przykład implementacji handlera aplikacji]
public async Task<CreateExpenseResponse> Handle(CreateExpenseCommand command, CancellationToken cancellationToken)
{
    var accountId = command.AccountId;
    var expenseDTO = _mapper.Map<AddExpenseDTO>(command);
    MoneyAccount moneyAccount = await _moneyAccountRepository.GetAsync(accountId);

    IReadOnlyExpense operation = moneyAccount.AddOperation(expenseDTO);

    _moneyAccountRepository.Update(moneyAccount);
    bool success = await _moneyAccountRepository.UnitOfWork.SaveEntitiesAsync();

    return success ? _mapper.Map<CreateExpenseResponse>(operation) : null;
}
\end{lstlisting}
}

\subsection{Szczegóły implementacji aplikacji klienckiej}
\label{subsec:szczegoly-implementacji-client}

Żądania użytkownika wysyłane do aplikacji klienckiej mapowane są do konkretnych stron Razor Pages. Strona Razor Page składa się z widoku napisanego przy pomocą języka html i opcjonalnie modelu strony (ang.~\emph{page model}). Na stronie często też znajdują się dodatkowe komendy składni Razor, których zadaniem jest budowanie widoku w zależności od modelu przekazanego do strony, a także generowanie widoków częściowych (ang.~\emph{partial views}). Widoki częściowe ułatwiają wydzielanie wspólnych części widoku strony do oddzielnych plików. Można łatwo wykorzystać je w różnych stronach Razor, a także użyć w pętli dla kolekcji obiektów zawartych w modelu widoku. Widoki częściowe nie mogą mieć swoich modeli strony, a jedynie polegają na danych przekazanych do nich z zewnątrz.

Aby widok był stroną Razor Page musi na nim znaleźć się dyrektywa \texttt{@page}. Służy ona także do routingu w aplikacji. Domyślnie routing oparty jest o strukturę folderów i nazwę strony. Oczywiście w takiej formie jest niewystarczający. Na listingu~\ref{list:razor-page-1} pokazano przykład wykorzystania dyrektywy \texttt{@page} do stworzenia strony konta bankowego o konkretnym identyfikatorze typu \texttt{guid} -- linia 1.

{\belowcaptionskip=-10pt
\begin{lstlisting}[label=list:razor-page-1,
    caption=Przykład strony Razor Page: \texttt{Edit.cshtml}]
@page "{accountId:guid?}"
@model BudgetApplication.Client.Pages.MoneyAccounts.EditModel
...
<h2>Edycja konta: @Model.MoneyAccount.Name</h2>
@if (Model.Message != null)
{
    <div class="alert alert-info">@Model.Message</div>
}
<form method="post">
  <input type="hidden" asp-for="MoneyAccount.Id" />

  <div class="form-group">
    <label asp-for="MoneyAccount.Name"></label>
    <input asp-for="MoneyAccount.Name" class="form-control" />
    <span asp-validation-for="MoneyAccount.Name" class="text-danger"></span>
  </div>
  ...
  <div class="form-group">
    <label asp-for="MoneyAccount.Type"></label>
    <select asp-for="MoneyAccount.Type" asp-items="Model.MoneyAccountTypes" class="form-control"></select>
    <span asp-validation-for="MoneyAccount.Type" class="text-danger"></span>
  </div>
  ...
</form>
@section Scripts {
    <partial name="_ValidationScriptsPartial" />
}
\end{lstlisting}
}

Na listingu~\ref{list:razor-page-1} pokazano także, w jaki sposób za pomocą składni Razor napełnić widok danymi z modelu strony. Przykładem takiej dyrektywy jest ta w linijce 4. Dyrektywy używające modelu strony mogą posłużyć także do warunkowego generowania części widoku -- linie 5-7.

Strona z listingu~\ref{list:razor-page-1} zawiera dyrektywę Razor \texttt{@model}. Określa ona typ modelu strony przypisanego do strony. Jego nazwa składa się z nazwy strony i doklejonego do niej przyrostka \texttt{Model}. Model strony jest ściśle powiązany ze stroną. Jest tzw. klasą ,,code-behind''. Pomiędzy widokiem strony Razor i modelem strony jest wiązanie, które sprawia, że zmiany na widoku odzwierciedlane są w modelu, a zmiany w modelu odzwierciedlane są na stronie. Ten proces ma miejsca poprzez synchroniczne żądania HTTP, w trakcie działań użytkownika, takich jak: wejście na stronę, odświeżenie strony, naciśnięcie guzika. Jest to realizacja popularnego wzorca architektonicznego \texttt{MVVM} (ang.~\emph{Model–view–viewmodel}).

Model strony to klasa dziedzicząca po klasie \texttt{PageModel}. Implementuje on dwie metody: \texttt{OnGetAsync} oraz \texttt{OnGetAsync}. W pierwszej z nich implementowana jest logika pobierania danych i napełniania nimi widoku strony. Druga z nich służy do obsługi żądań HTTP POST użytkownika, a w szczególności do wysyłania formularzy. 

Na listingu~\ref{list:razor-page-model-1} przedstawiono przykład modelu strony. Zawiera on właściwość typu \texttt{MoneyAccount}. Pola jej klasy są bezpośrednio mapowane na wartości za pomocą dyrektyw Razor w pliku strony -- listing~\ref{list:razor-page-1}. W metodzie \texttt{OnGetAsync} przypisywany jest obiekt konta bankowego do tej właściwości. Warto zwrócić uwagę, że zarówno parametr metody \texttt{OnGetAsync}, jak i parametr dyrektywy routingu \texttt{@page} (listing~\ref{list:razor-page-1} linia 1) mają typ mogący być wartością \texttt{null}. Dzięki temu strona obsługuje dwie ścieżki routingu: \texttt{/Edit/{id}} do strony edycji istniejącej konta bankowego o podanym id oraz \texttt{/Edit} do strony tworzenia nowego konta bankowego. Podobna sytuacja ma miejsce w metodzie \texttt{OnPostASync}.

Atrybut \texttt{[BindProperty]} dla właściwości \texttt{MoneyAccount} sprawia, że konto bankowe tworzone jest na podstawie danych z żądania i od razu przypisywane do właściwości, dlatego metoda \texttt{OnPostAsync} nie musi mieć parametru typu \texttt{MoneyAccount}. Domyślnie ten atrybut działa tylko dla żądania POST, jednak łatwo go skonfigurować dla żądań GET. W tym przykładzie jest to niepotrzebne.

Za pomocą metody \texttt{GetEnumSelectList} tzw.\ ,,Tag Helper'', łatwo zmapowowano wartości enumeratora \texttt{MoneyAccountType} na kolekcję, wykorzystywaną w rozwijanym menu w formularzu na listingu~\ref{list:razor-page-1} - linia 18-21.

{\belowcaptionskip=-10pt
\begin{lstlisting}[label=list:razor-page-model-1,
    caption=Przykład modelu strony Razor Page: \texttt{EditPage.cs}]
public class EditModel : PageModel
{
...
  [BindProperty]
  public MoneyAccount MoneyAccount { get; set; }
  public IEnumerable<SelectListItem> MoneyAccountTypes { get; set; }
  public async Task<IActionResult> OnGetAsync(Guid? accountId)
  {
    MoneyAccountTypes = _htmlHelper.GetEnumSelectList<MoneyAccountType>();
    if (accountId.HasValue)
      MoneyAccount = await _accountHttpClient.GetMoneyAccountByIdAsync(accountId.Value);
    else
      MoneyAccount = new MoneyAccount();
    
    return MoneyAccount == null 
      ? RedirectToPage("./NotFound")
      : Page() as IActionResult;
  }

  public async Task<IActionResult> OnPostAsync()
  {
    if (!ModelState.IsValid)
    {
      MoneyAccountTypes = _htmlHelper.GetEnumSelectList<MoneyAccountType>();
      return Page();
    }
    try
    {
      MoneyAccount = MoneyAccount.Id == Guid.Empty 
        ? await _accountHttpClient.AddMoneyAccountsAsync(MoneyAccount)
        : await _accountHttpClient.EditMoneyAccountsAsync(MoneyAccount);

      return RedirectToPage("./Edit", new { accountId = MoneyAccount.Id });
    }
    ...
  }
}
\end{lstlisting}
}

Na końcu listingu~\ref{list:razor-page-1} znajduje się dyrektywa generująca widok częściowy w sekcji \texttt{Scripts} głównego układu strony -- plik \texttt{\_Layout.cshtml}. Odpowiada ona za generowanie na stronie \texttt{Edit.cshtml} skryptów biblioteki \texttt{jQuery unobtrusive validation}, która współpracuje z~frameworkiem ASP.NET Core i zapewnia walidację po stronie klienta na podstawie adnotacji modelu klasy języka C\#. 

Przykład walidowanej klasy znajduje się na listingu~\ref{list:razor-validation-1}. Przed tagiem \texttt{body} HTML w~pliku \texttt{\_Layout.cshtml} znajduje się dyrektywa \texttt{@RenderSection("{}Scripts", required: false)}. Umożliwia dołączenie opcjonalnej sekcji skryptów przez widoki, korzystające z tego układu strony.


{\belowcaptionskip=-10pt
\begin{lstlisting}[label=list:razor-validation-1,
    caption=Przykład klasy z adnotacjami]
using System.ComponentModel.DataAnnotations;
public class MoneyAccount
{
    public Guid Id { get; set; }
    [Required]
    public Guid UserId { get; set; }
    [Required, MinLength(8)]
    public string Name { get; set; }
    [Required]
    public MoneyAccountType Type { get; set; }
    [MaxLength(255)]
    public string Description { get; set; }
...
    [MinLength(26)]
    [MaxLength(26)]
    public string BankAccountNumber { get; set; }
}
\end{lstlisting}
}

\subsection{Strony aplikacji klienckiej}
\label{subsec:widok-client}

W tym punkcie przedstawione są niektóre widoki z aplikacji klienckiej. Na pierwszym z nich -- rysunek~\ref{fig:ss-6}, przedstawiono widok listy kont pieniężnych użytkownika. Oferuję on możliwość dodanie nowego konta, jak i również wykonania akcji edycji, usunięcia i przejrzenia informacji o wybranym koncie.

\begin{figure}[ht]
	\centering
\includegraphics[scale=.35]{rys04/lista-kont.PNG}
	\caption{Widok listy kont użytkownika}
	\label{fig:ss-6}
\end{figure}

Kolejny widok~\ref{fig:ss-1} przedstawia ekran, który pojawia się po naciśnięciu guzika ,,Dodaj nowe konto''. Pojawiają się na nim formularz do wypełnienia przez użytkownika. Aby dodać nowe konto użytkownik musi zatwierdzić dane naciskając guzik ,,Zapisz''.

\begin{figure}[ht]
	\centering
	\includegraphics[scale=.35]{rys04/dodaj-konto.PNG}
	\caption{Dodawanie nowego konta}
	\label{fig:ss-1}
\end{figure}

Po dodaniu konta użytkownik znajduje się dalej na tym samym widoku, jednak nie jest to już strona dodania nowego konta, a edycji tego nowostworzonego. Można to zauważyć po zmianie komunikatu z ,,Dodawanie nowego konta'' na ,,Edycja konta: \{nazwa\}''. Pojawia się też jednorazowy komunikat ,,Konto \{nazwa\} zapisano'' -- rysunek~\ref{fig:ss-2}. Znika on po odświeżeniu strony.

\begin{figure}[ht]
	\centering
	\includegraphics[scale=.35]{rys04/konto-saved.PNG}
	\caption{Edycja stworzonego konta}
	\label{fig:ss-2}
\end{figure}

Z widoku listy kont można także przejść do widoku informacji o koncie -- rysunek~\ref{fig:ss-3}. Na~tym widoku zaprezentowane są informacje o koncie przeznaczone dla użytkownika bez prawa edycji. Guzik powrotu do ekranu listy kont ułatwia sprawną nawigację w aplikacji.

\begin{figure}[ht]
	\centering
	\includegraphics[scale=.35]{rys04/konto-info.PNG}
	\caption{Widok informacji o koncie}
	\label{fig:ss-3}
\end{figure}

Na rysunku~\ref{fig:ss-4} zaprezentowano ekran, pojawiający się po naciśnięciu guzika usuń -- ikonka kosza na śmieci, na stronie z listą kont. Po zatwierdzeniu usunięcia konta użytkownik jest przekierowywany do listy kont. Pojawia się także komunikat o usunięciu strony -- rysunek~\ref{fig:ss-5}.

\begin{figure}[ht]
	\centering
\includegraphics[scale=.35]{rys04/usun-konto.PNG}
	\caption{Usuwanie konta}
	\label{fig:ss-4}
\end{figure}

\begin{figure}[ht]
	\centering
\includegraphics[scale=.35]{rys04/usunieto.PNG}
	\caption{Komunikat o usunięciu konta}
	\label{fig:ss-5}
\end{figure}