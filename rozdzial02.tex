\chapter{Przygotowania \emph{Know-how}}
\label{chap:know-how}
\section{Projektowanie REST API}
\label{sec:projektowanie-api}
Projektowanie i tworzenie \texttt{API} (ang.~\emph{Application Programming Interface}) współcześnie jest jednym z podstawowych zadań programisty. Zadaniem API jest udostępnienie i ochrona swoich zasobów. Zasoby te to najczęściej dane i w zależności od uprawnień klient może je zmieniać, lub tylko pobierać. Klientem może być strona internetowa służąca do wyświetlania i zarządzania danymi, a także inna aplikacja serwerowa korzystająca z tych danych.

Użycie API do dostarczania danych do aplikacji klienckiej pozwala rozdzielić logikę widoku po stronie klienta oraz logikę biznesową po stronie serwera. Zaletą tego rozwiązania jest uproszczenie implementacji i w perspektywie ułatwienia utrzymywania tworzonego programu. Ponadto tak tworzone oprogramowanie jest bardziej elastyczne i łatwiej można je integrować z~innymi systemami. Stworzenie takiego API nie jest zadaniem łatwym. Dlatego w sekcji~\ref{subsec:api-dobre-praktyki} przedstawione zostaną dobre praktyki.

Często trzeba szybko stworzyć nowe API lub zmodyfikować już istniejące. Dodawanie funkcji filtrowania, sortowania czy łączenia danych z~różnych zasobów jest czasochłonne i~nudne. Implementacja takich funkcjonalności zmusza nas do dodawania nowych punktów dostępu (ang.~\emph{endpoint}) do już istniejących zasobów lub modyfikowania tych już istniejących. Problemem jest także budowanie adresu \texttt{URL} (ang.~\emph{Uniform Resource Locator}). Musi być on elastyczny i intuicyjny, aby można było pracować bez ciągłego wertowania dokumentacji. Z pomocą tutaj przychodzi zapoczątkowany przez Microsoft i zatwierdzony przez \texttt{ISO} (ang.~\emph{International Organization for Standardization}) protokół \texttt{OData} (ang.~\emph{Open Data Protocol}) i jego gotowe implementacje. Zostanie on opisany w sekcji~\ref{subsec:odata}

\subsection{Dobre praktyki tworzenia API} 
\label{subsec:api-dobre-praktyki}
Aby stworzyć czytelne i elastyczne REST API należy zastosować się do wielu zasad. Oczywiście nie są one obowiązkowe. Najczęściej jednak stosowanie się do nich ułatwia konsumowanie API i integrowanie go z innymi systemami. Jednym z koniecznych do szybkiej i bezproblemowej pracy z API kroków jest stworzenie i późniejsze aktualizowanie jego dokumentacji. Polecanym i nowoczesnym narzędziem służący do jej tworzenia jest \texttt{Swagger}. 

API projektujemy dla zasobów, które są rzeczownikami. Dlatego adres URL, a dokładniej jego ścieżka zasobów (ang.~\emph{resource path}) powinna być złożona z rzeczowników. Zalecane jest, aby używać liczby mnogiej rzeczownika, gdyż zazwyczaj zasób to zbiór przykładowo kont bankowych. Natomiast, aby wyrazić akcję, którą wykonuje się na zasobie należy użyć odpowiedniej metody \texttt{HTTP} (ang.~\emph{HTTP Verb}). W tabeli~\ref{tab:http-verb} widać wzorzec zastosowania metod HTTP~\cite{api-good-practises-1}.

\begin{table}
    \centering
    \caption{Użycie metod HTTP}
    \label{tab:http-verb}
    \begin{tcolorbox}[tab2,tabularx={X||Y|Y|Y|Y}]
    \texttt{Zasób}      & \texttt{GET}     & \texttt{POST}    & \texttt{PUT}      & \texttt{DELETE}   \\\hline\hline
    /bank-account   & Zwraca listę kont & Tworzy nowe konto &  Zbiorczo aktualizuje konta &  Usuwa wszystkie konta \\\hline
    {\small /bank-account/1} & Zwraca określone konto & Metoda niedozwolona (405) &  Aktualizuje określone konto &  Usuwa określone konto \\\hline
    \end{tcolorbox}
\end{table} 

Kolejną ważną praktyką przy tworzeniu API jest zwracanie odpowiedniego kodu odpowiedzi HTTP (ang.~\emph{HTTP Status Code}), aby poinformować klienta o sposobie realizacji jego żądania. Żądanie może zakończone być sukcesem bądź niepowodzeniem. Jednak konieczne jest używanie więcej niż dwóch kodów odpowiedzi, aby dokładnie określić, co ten sukces lub niepowodzenie oznaczają. Dokument \texttt{RFC7231} opisuje, kiedy i jak używać konkretnych kodów odpowiedzi HTTP. Statusy HTTP dzielimy na pięć klas, które rozróżniane są po pierwszej cyfrze kodu odpowiedzi HTTP~\cite{rfc7231}.

\begin{itemize}
\item \texttt{1XX} -- Informacja -- Żądanie zostało odebrane, proces jest kontynuowany.  
\item \texttt{2XX} -- Sukces -- Żądanie zostało pomyślnie odebrane, zrozumiane i zaakceptowane. 
\item \texttt{3XX} -- Przekierowanie -- Konieczne jest podjęcie dalszych działań w celu zakończenia żądania.
\item \texttt{4XX} -- Błąd klienta -- Żądanie zawiera złą składnię lub nie może być zrealizowane.
\item \texttt{5XX} -- Błąd serwera -- Serwerowi nie udało zrealizować się poprawnego żądania.
\end{itemize}

W tabeli~\ref{tab:http-status} przedstawiono najczęściej używane w REST API statusy HTTP i ich znaczenie~\cite{api-good-practises-2, rfc7231}.  
Warto wspomnieć, że statusu \texttt{204} używa się najczęściej, gdy serwer nie zwraca w~ciele odpowiedzi żadnych danych. Sytuacją taką jest między innymi usuwanie zasobu.

\begin{table}
    \centering
    \caption{Kody odpowiedzi HTTP}
    \label{tab:http-status}
    \begin{tcolorbox}[tab2,tabularx={p{.30\linewidth}|Y}]
    \texttt{Kod HTTP}    & \texttt{Opis}   \\\hline\hline
    \texttt{200: OK}
        & Żądanie zakończone sukcesem. \\\hline
    \texttt{201: Created}
        & Stworzony został nowy zasób. \\\hline
    \texttt{204: No Content}
        & W ciele odpowiedzi nie ma żadnej dodatkowej treści.\\\hline
    \texttt{400: Bad Request}
        & Serwer nie może przetworzyć żądania z powodu błędu klienta.   \\\hline
    \texttt{401: Unauthorized}
        & Żądanie nie zostało przetworzone, ponieważ brakuje poprawnych danych uwierzytelniających dla docelowego zasobu. \\\hline
    \texttt{403: Forbidden}
        & Podane dane uwierzytelniające nie są wystarczające do autoryzacji żądania. \\\hline
    \texttt{404: Not Found}
        & Serwer źródłowy nie znalazł bieżącej reprezentacji zasobu docelowego lub nie chce ujawnić, że istnieje. \\\hline
    \texttt{500: Internal Server Error}
        & Serwer napotkał nieoczekiwany warunek, który uniemożliwił mu spełnienie żądania. \\\hline
    \end{tcolorbox}
\end{table}

Ważną zasadą dotyczącą API jest to, aby po stworzeniu nowego zasobu zwróciła jego reprezentację, na przykład w postaci \texttt{JSON} (ang.~\emph{JavaScript Object Notation}). Dotyczy to w szczególności metody \texttt{POST}. W przypadku powodzenia serwer powinien zwrócić w takim wypadku odpowiedź z kodem \texttt{201}, a także dodatkowy nagłówek z lokalizacją nowostworzonego zasobu (ang.~\emph{Location Header}). Dodatkowo rekomendowane jest, aby dane stworzonego obiektu lub obiektów zapakować w pole danych (ang.~\emph{data} co~pokazano na listingu~\ref{list:response200}. 

{\belowcaptionskip=-10pt
\begin{lstlisting}[label=list:response200,
    caption=Przykład pomyślnej odpowiedzi serwera]
//200 OK
{
  "errors": [],
  "data": {
    "firstName": "Paweł",
    "lastName": "Żółaniecki",
    "age": 24
  },
}
\end{lstlisting}
}

Umieszczanie lokalizacji dostępnych zasobów wpisuję się w realizację koncepcji \texttt{HATEOAS} (ang.~\emph{Hypermedia As Transfer Engine Of Application State}). Pełna realizacja tej koncepcji zapewnia łatwą nawigację pomiędzy powiązanymi zasobami i przedstawia dostępne akcje na danym zasobie, poprzez umieszczanie w metadanych odpowiedzi linków URL do konkretnych akcji i zasobów.

W zależności od strategii biznesowej przyjętej w projekcie, gdy klient nie ma uprawnień do~zasobu, można zwrócić nie tylko kod \texttt{403}, ale także \texttt{404}, gdy serwer chce ukryć istnienie zasobu.

Dobrym zwyczajem jest umieszczanie w odpowiedzi z kodem \texttt{400} i \texttt{403} dokładniejszej informacji o przyczynie braku realizacji żądania. W podobny sposób powinno się informować klienta także o błędach. Częstą praktyką jest zwracanie w ciele odpowiedzi tabeli z listą błędów~\ref{list:response400}

{\belowcaptionskip=-10pt
\begin{lstlisting}[label=list:response400,
    caption=Odpowiedź serwera zawierająca opis błędu]
//400 Bad Request
{
  "errors": [
    {
      "status": 400,
      "details": "Invalid state. Valid state are 'draft', 'success' or 'failure'"
    }
  ]
}
\end{lstlisting}
}

Istotnym także jest, żeby nie tworzyć nowych adresów \texttt{URL} dla filtrowanego czy sortowanego zasobu. Lepszym podejściem jest przedstawić wyżej wymienione kryteria w ścieżce wyszukiwania adresu \texttt{URL} (ang.~\emph{query string}). W sekcji~\ref{subsec:odata} przedstawiony zostanie wybrany przez autora elastyczny protokół, określający strategię wyszukiwania i filtrowania zasobów za pomocą określonych parametrów ścieżki wyszukiwania adresu URL.

\subsection{Protokół OData}
\label{subsec:odata}

OData jest protokołem, definiującym nowe standardy i~praktyki budowania i konsumowania REST API. Biblioteki implementujące ten standard niejako zwalniają programistę z dbania o konwencje adresów URL, nagłówki żądań i kody odpowiedzi HTTP. OData wykorzystywany jest w produktach firm takich jak: Microsoft, IBM, Telerik czy SAP.

OData i jego implementacje świetnie nadają się do szybkiego tworzenia REST API, a na pewno już tej jego części, która odpowiada za przetwarzanie zapytań (ang.~\emph{queries}). Zapytania są to operacje, które nie modyfikują stanu systemu, a jedynie zwracają dane, o które pyta klient. Zazwyczaj wykonywane są za pomocą metody HTTP GET. 

Charakterystyką zapytań jest to, że najczęściej klient nie potrzebuje wszystkich danych jakie są w systemie. Często na danych chce się wykonać projekcję, filtrowanie czy sortowanie. Nierzadko też potrzeba dokonać złączenia danych jednego zasobu z danymi innego zasobu. 

Filtrowanie i sortowanie to operacje, które często implementuje się w API z wykorzystaniem ścieżki wyszukiwania adresu URL. Operacja na listingu~\ref{list:odata-projekcja-sortowanie} pobiera imiona i nazwiska  wszystkich ludzi, posortowane malejące według ich wieku.


\begin{lstlisting}[label=list:odata-projekcja-sortowanie,
    caption=OData -- przykład projekcji i sortowania]
GET http://localhost:44300/api/odata/Person?$select=firstName, lastName$orderby=age desc
\end{lstlisting}

O ile jest to zadanie proste, lecz żmudne, o tyle przy projektowaniu złączenia danych zasobów, programista mając do wykonania owo zadanie, często stworzy w API całkiem nowy zasób z nowym adresem URL. W większości takich przypadków dobrym rozwiązaniem jest użycie biblioteki implementującej protokół OData, która pozwala zautomatyzować wymagane funkcje.


\section{Użyte wzorce projektowe}
\label{sec:wzorce}

\subsection{Mediator}
\label{subsec:mediator}

\subsection{Podział odpowiedzialności czyli CQRS}
\label{subsec:cqrs}

\begin{table}[htb]
\centering\small
\caption{Pliki źródłowe szablonu oraz wyniki kompilacji}
\label{tab:szablon}
\begin{tabularx}{\linewidth}{|p{.55\linewidth}|X|}\hline
Źródła & Wyniki kompilacji \\ \hline\hline
\verb?Dokument.tex? - dokument główny\newline
\verb?Dokument.tcp? -- szablon projektu \texttt{MiKTeX}\newline
\verb?rozdzial01.tex? -- plik rozdziału \texttt{01}\newline
\verb?...?\newline
\verb?dodatekA.tex? -- plik dodatku \texttt{A}\newline
\verb?...?\newline
\verb?rys01? -- katalog na rysunki do rozdziału \texttt{01}\newline
\verb?   |- fig01.png? -- plik grafiki\newline
\verb?   |- ...?\newline
\verb?...?\newline
\verb?rysA? -- katalog na rysunki do dodatku \texttt{A}\newline
\verb?   |- fig01.png? -- plik grafiki\newline
\verb?   |- ...?\newline
\verb?...?\newline
\verb?dokumentacja.bib? -- plik danych bibliograficznych\newline
\verb?Dyplom.ist? -- plik ze stylem indeksu\newline
\verb?by-nc-sa.png? -- plik z ikonami CC\newline
 &
\verb?Dyplom.bbl?\newline
\verb?Dyplom.blg?\newline
\verb?Dyplom.ind?\newline
\verb?Dyplom.idx?\newline
\verb?Dyplom.lof?\newline
\verb?Dyplom.log?\newline
\verb?Dyplom.lot?\newline
\verb?Dyplom.out?\newline
\verb?Dyplom.pdf? -- dokument wynikowy\newline
\verb?Dyplom.syntex?\newline
\verb?Dyplom.toc?\newline
\verb?Dyplom.tps?\newline
\verb?*.aux?\newline 
\verb?Dyplom.synctex?\newline\\
\hline
\end{tabularx}
\end{table}


\begin{lstlisting}[basicstyle=\ttfamily]
> pdflatex Dyplom.tex
\end{lstlisting}

\begin{lstlisting}[basicstyle=\ttfamily]
> bibtex Dyplom
\end{lstlisting}
Szczegóły dotyczące przygotowania danych bibliograficznych oraz zastosowania cytowań przedstawiono w podrozdziale \ref{sec:literatura}.

W głównym pliku zamieszczono polecenia pozwalające sterować procesem kompilacji poprzez włączanie bądź wyłączanie kodu źródłowego poszczególnych rozdziałów. Włączanie kodu do kompilacji zapewniają instrukcje \verb+\include+ oraz \verb+\includeonly+. Pierwsza z nich pozwala włączyć do kompilacji kod wskazanego pliku (np.\ kodu źródłowego pierwszego rozdziału \verb+\chapter{Wstęp}
\label{chap:wstep}
\section{Wprowadzenie}
\label{sec:wprowadzenie}
Zarządzanie finansami to jeden z ważniejszych aspektów funkcjonowania każdego przedsiębiorstwa, bez względu na rodzaj prowadzonej działalności, a także liczbę pracowników. Rozliczanie przychodów i wydatków jest nie tylko bardzo często obowiązkiem, ale również pomocą w panowaniu nad przepływem środków finansowych. 

Jako szczególny typ przedsiębiorstwa można uznać gospodarstwo domowe, czyli po prostu rodzinę. Z ekonomicznego punktu widzenia jej celem jest zaspokajanie wspólnych i osobistych potrzeb. Rodzina jest też zespołem osób, które wspólnie podejmują decyzję o swoim podstawowym utrzymaniu i zarządzaniu wniesionymi do budżetu domowego środkami. Gospodarstwem domowym można także nazwać pojedynczą osobę utrzymującą się samodzielnie~\cite{gospodarstwo-domowe}.

W dobie rosnącej konsumpcji oraz na skutek celowego stosowania różnych technik pobudzania zapotrzebowania przez producentów panowanie nad wydatkami zaczyna sprawiać sporo trudności. Problemem staje się odróżnienie zachcianek od tego, co z punktu widzenia funkcjonowania rodziny jest niezbędne. W konsekwencji dość łatwo jest doprowadzić do załamania całego domowego budżetu.

Z panowaniem nad wydatkami bezpośrednio wiąże się monitorowanie dochodów i~inwestycji gospodarstwa domowego. Oprócz tradycyjnych dochodów z tytułu etatu, rodzina może mieć inne źródła przychodu, jak na przykład praca dorywcza czy wynajem nieruchomości. Nieruchomości to niejedyna możliwość ulokowania oszczędności gospodarstwa domowego. Innymi przykładami inwestycji mogą być lokaty, obligacje skarbowe, złoto, a nawet akcje giełdowe. Coraz częściej jednym ze sposobów ich zaoszczędzenia jest dołączanie do programów lojalnościowych, które wiążą się dodatkowymi rabatami oraz bonami na zakupy. Nierzadko też bogata oferta banków pozwala na odłożenie dodatkowej gotówki czy zbieranie punktów w podobnych programach. W~obu przypadkach warunkiem uczestnictwa w promocji jest wyrobienie dodatkowej karty bądź pobranie i rejestracja w aplikacji mobilnej banku lub sprzedawcy.

Kolejnym ważnym aspektem domowych finansów jest trwałość zakupionych towarów, takich, jak na przykład ubrania oraz sprzęt RTV i AGD. Rozklejające się po kilku miesiącach buty, psująca się lodówka, jak również innego rodzaju uszkodzenia i usterki, mogą prowadzić do nadwyrężenia budżetu domowego. Dlatego w wielu domach trzyma się paragony i faktury w celu dokonania potencjalnej reklamacji. Spotykane są rozwiązania umożliwiające zarówno powiązania zakupów z kontem w aplikacji lojalnościowej, jak i dedykowane programy do trzymania dokumentów sprzedażowych w formie elektronicznej.

% TO DO: proszę dołożyć cytowania (do stron domowych wymienonych niżej aplikacji).
Popularną aplikacją, która pozwala na elektroniczne przechowywanie paragonów jest ,,Pan Paragon''. Umożliwia on nie tylko na trzymanie dokumentów sprzedażowych w pamięci lokalnej telefonu, ale również na ich archiwizację w chmurze. Pozwala on także na przechowywanie kart lojalnościowych. Niestety, nie daje możliwości do zarządzania wydatkami i planowania budżetu. Jest też aplikacją dedykowaną jednostce, a nie rodzinie.

Istnieje też wiele programów do zarządzania budżetem domowym. Jednym z przykładów jest ,,Wallet - przychody i wydatki, karty lojalnościowe''. Niestety, zakres oferowanych w nim funkcji jest ograniczony. Użytkownik w darmowej wersji może zdefiniować tylko trzy konta. Nie ma też możliwości integrowania wydatków innych członków rodziny, co komplikuje sprawny nadzór nad całym domowym budżetem.

W ostatnim czasie również supermarkety czy sklepy odzieżowe wypuszczają własne oprogramowanie, umożliwiające branie udziału w programach lojalnościowych. Jednym z nich jest Lidl, którego rozwiązanie oferuje też zapisanie paragonu w aplikacji. Cały proces odbywa się poprzez zeskanowanie kodu QR (ang.~\emph{quick response}) w telefonie. W przypadku tego typu programów, problemem staje się mnogość rozwiązań dla pojedynczych sieci i marek.

Jak widać dbanie o budżet gospodarstwa domowego nie jest rzeczą trywialną. Wielowymiarowość tego zadania może przyprawić osobę odpowiedzialną o zawrót głowy. Dużym utrudnieniem jest brak możliwości synchronizowania działań całej rodziny w ramach jednego systemu. Ponadto nie wystarczy tylko panować nad przychodami i wydatkami. Trzeba wziąć pod uwagę czynniki, takie jak zabezpieczenie się przed utratą środków z powodu przedwczesnego zużycia się produktów, a także  programy lojalnościowe, które oferują atrakcyjne promocje. Tutaj sprawę komplikuje ilość potencjalnych kart i możliwość współdzielenia ich między członkami rodziny. 

W tej pracy dyplomowej zdecydowano się przedstawić projekt aplikacji budżetu domowego, która łączyć będzie wszystkie wyżej wymienione funkcje. Aplikacja ma stać się uniwersalnym narzędziem do zarządzania finansami całej rodziny, ze szczególnym uwzględnieniem roli rodziców i dzieci. Ważnymi funkcjami projektu mają być również te związane z tworzeniem kopii zapasowej paragonów, a także zarządzaniem gwarancjami i mnogością kart lojalnościowych.

\section{Cel i zakres pracy}
\label{sec:cel-zakres}
\subsection{Cel}
\label{subsec:cel}
Celem pracy jest stworzenie aplikacji do zarządzania budżetem domowym, oferującej użytkownikowi bezpieczne i płynne korzystanie z dostarczanych w niej funkcji.
Aplikacja ta w założeniach ma być produktem MVP (ang.~\emph{minimum valuable product}), czyli rozwiązaniem zapewniającym minimalną wartość użytkową.  

Z realizacją celu wiąże się opracowanie koncepcji uniwersalnej aplikacji dla domowego budżetu, która łączyłaby w sobie funkcje innych programów tego typu z~funkcjami programów przechowujących kopie paragonów, a także oferujących możliwość przechowywania kart lojalnościowych, bonów i gwarancji. 
Ponadto w pracy mają powstać założenia projektu elastycznego systemu o nowoczesnej architekturze, umożliwiającej łatwe dodawanie nowych funkcji i integrację z zewnętrznymi dostawcami usług. 

\subsection{Zakres}
\label{subsec:zakres}
W ramach pracy zbudowane zostaną dwie aplikacje webowe. Pierwsza z nich ma oferować graficzny interfejs użytkownika (ang.~\emph{graphical user interface}, GUI) wyświetlany w oknie przeglądarki internetowej, druga zaś ma dostarczyć interfejs programistyczny aplikacji (ang.~\emph{application programming interface}, API)

Zaimplementowane funkcje powinny umożliwiać tworzenie i zarządzania  rachunkiem, tudzież kontem pieniężnym, wraz z możliwością tworzenia wydatków i przychodów. Dodatkowo obsługiwane mają być: tworzenie i przypisywanie kategorii operacji pieniężnych, jak również bezpieczny i nowoczesny sposób rejestracji oraz logowania się użytkownika w systemie, zaimplementowany z wykorzystaniem mechanizmu tokenów.

Aplikacja zostanie dostarczona na środowisko produkcyjne stworzone na platformie Azure. Zostanie ona przetestowana w środowisku deweloperskim oraz produkcyjnym. Ponadto zostanie spisana dokumentacja API.

\section{Układ pracy}
\label{sec:uklad-pracy}
W rozdziale~\ref{chap:wstep} przedstawiono współczesne problemy związane z prowadzeniem domowych finansów i uzasadnienie konieczności stworzenia aplikacji budżetu domowego. Omówiono w~nim również cel pracy inżynierskiej oraz zakres wykonanych prac.

W rozdziale \ref{chap:zalozenia-projektowe} przedstawiono założenia projektowe wraz z~zarysem architektury systemu i~poszczególnych jego komponentów. Zawarto w nim także wymagania funkcjonalne i niefunkcjonalne aplikacji.

Rozdział~\ref{chap:know-how} poświęcono na wstęp teoretyczny. Opisano w nim bazę wiedzy zgromadzoną przez autora podczas przygotowań do realizacji pracy. W podrozdziale~\ref{sec:projektowanie-api} skupiono uwagę na~dobrych praktykach tworzenia API~oraz przydatnych narzędziach. W podrozdziale~\ref{sec:wzorce} przybliżono wykorzystane w~kodzie wzorce projektowe. W podrozdziale~\ref{sec:autoryzacja} dokonano prezentacji użytego mechanizmu autoryzacji i uwierzytelniania.

Szczegóły implementacji opisano w~rozdziale~\ref{chap:implementacja}. W podrozdziale~\ref{sec:struktura-projektu} przedstawiono fizyczną i logiczną strukturę projektu, a~w podrozdziale~\ref{sec:szczegoly-implementacji} -- detale realizacji aplikacji klienckiej (ang.~\emph{frontend}) oraz API działającego po stronie serwera (ang.~\emph{backend}).

Na omówienie testów API i aplikacji klienckiej poświęcono rozdział~\ref{chap:testy}. 

W ostatnim rozdziale~\ref{chap:podsumowanie} zamieszczono podsumowanie pracy wraz z uwagami odnośnie zdobytych doświadczeń.+). Druga, jeśli zostanie zastosowana, pozwala określić, które z~plików zostaną skompilowane w całości (na przykład kod źródłowy pierwszego i drugiego rozdziału \verb+\includeonly{rozdzial01.tex,rozdzial02.tex}+).
Brak nazwy pliku na liście w poleceniu \verb+\includeonly+ przy jednoczesnym wystąpieniu jego nazwy w poleceniu \verb+\include+ oznacza, że w kompilacji zostaną uwzględnione referencje wygenerowane dla tego pliku wcześniej, sam zaś kod źródłowy pliku nie będzie kompilowany. 

W szablonie wykorzystano klasę dokumentu \texttt{memoir} oraz wybrane pakiety. Podczas kompilacji szablonu w \texttt{MikTeXu} wszelkie potrzebne pakiety zostaną zainstalowane automatycznie (jeśli \texttt{MikTeX} zainstalowano z opcją dynamicznej instalacji brakujących pakietów). W przypadku innych dystrybucji latexowych może okazać się, że pakiety te trzeba doinstalować ręcznie (np.\ pod linuxem z \texttt{TeXLive} trzeba doinstalować dodatkową zbiorczą paczkę, a jeśli ma się menadżera pakietów latexowych - to pakiety latexowe można instalować indywidualnie).

Jeśli w szablonie będzie wykorzystany indeks rzeczowy, kompilację źródeł trzeba będzie rozszerzyć o kroki potrzebne na wygenerowanie plików pośrednich \texttt{Dokument.idx} oraz \texttt{Dokument.ind} oraz dołączenia ich do finalnego dokumentu (podobnie jak to ma miejsce przy generowaniu wykazu literatury).
Szczegóły dotyczące generowania indeksu rzeczowego opisano w podrozdziale~\ref{sec:indeks}.

\section{Autoryzacja OpenId Connect}
\label{sec:autoryzacja}