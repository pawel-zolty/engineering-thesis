\chapter{Podsumowanie}
\label{chap:podsumowanie}

Celem pracy było zbudowanie systemu do zarządzania domowym budżetem o minimalnej wartości użytkowej. System miał umożliwiać użytkownikowi sprawne zarządzanie wydatkami i przychodami, w szczególności z podziałem na wiele kont bankowych.

System miał składać się z dwóch aplikacji webowych: pierwszej klienckiej -- dostarczającej interfejs użytkownika w przeglądarce oraz API -- gwarantującego szybkie i poprawne, z biznesowego punktu widzenia, przetwarzanie żądań aplikacji klienckiej. Obie te aplikacje zostały stworzone i są gotowe do dalszego rozwoju oraz działania w środowisku produkcyjnym. 

Funkcjonalności dotyczące zarządzania kontami pieniężnymi i środkami finansowymi w ramach rodziny, a zdefiniowane w zakresie pracy, zostały szczegółowo przeanalizowane i opisane z wykorzystaniem technik inżynierii oprogramowania oraz języka UML. Autor szczegółowo opisał ich przypadki użycia, dzieląc system na cztery mniejsze podsystemy.

Funkcjonalności obsługi konta bankowego i manipulowania jego środkami finansowymi zostały zaimplementowane dla indywidualnego użytkownika. Użytkownik może dodawać konta, a także operacje finansowe po zalogowaniu się do systemu.

Funkcjonalności dotyczące obsługi całej rodziny zostały zaprojektowane i opisane. Jednak zgodnie z zakresem pracy nie zostały zaimplementowane.

Niestety aplikacja kliencka od strony graficznej nie została wykonana w sposób zadowalający. Spełnia ona swoje wymagania funkcjonalne, jednak jej szata graficzna korzysta z domyślnego ostylowania, zapewnionego przez środowisko programistyczne Visual Studio.

Przy okazji tworzenia pracy autor zapoznał się z wieloma wzorcami projektowymi, takimi jak mediator, \emph{Command Query Responsibility Segregation} cz \emph{Model–view–viewmodel}. Autor zgłębił także dobre praktyki tworzenia oprogramowania, a w szczególności interfejsu programistycznego API. Autor wykorzystał zdobytą wiedzę do stworzenia elastycznego oprogramowania gotowego do dalszego rozwoju w profesjonalnym środowisku deweloperskim. Oprogramowanie jest także stworzone z wykorzystaniem najlepszych praktyk sztuki zawodowej, które ułatwiają utrzymanie aplikacji.

Autor przetestował także aplikację kliencką oraz API, tak aby zapewnić najwyższą jakość dostarczanego oprogramowania i świadczonych przez nie usług. Testy zostały wykonany w sposób rzetelny i dokładny. Testy potwierdziły dobrą jakość stworzonego oprogramowania i gotowość używania go w środowisku produkcyjnym.

Aplikacja po zaimplementowaniu reszty funkcjonalności, spełniać będzie założenia postawione przez autora i stanie się uniwersalnym narzędziem do zarządzania domowym budżetem, ze szczególną kontrolą administratora rodziny.

W przyszłości aplikacja może zostać poszerzona o funkcje związane z generowaniem wydatków na podstawie zdjęć paragonów, a także tworzenie ich kopii zapasowej w celu zarządzania gwarancjami produktów.

W aplikacji mogą również znaleźć się funkcje umożliwiające przechowywanie kart lojalnościowych, bonów i innych dokumentów sprzedażowych.