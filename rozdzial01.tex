\chapter{Wstęp}
\section{Wprowadzenie}
Bez względu na rodzaj działalności, a także liczbę pracowników, zarządzanie finansami to jeden z ważniejszych aspektów funkcjonowania każdego przedsiębiorstwa. Rozliczanie przychodów i wydatków jest nie tylko bardzo często obowiązkiem, ale zarówno pomocą w panowaniu nad przepływem pieniędzy wynikających z prowadzenia działalności gospodarczej. Jako szczególny typ przedsiębiorstwa można podać gospodarstwo domowe, czyli po prostu rodzinę.

Z ekonomicznego punktu widzenia, gospodarstwo domowe to jednostka gospodarująca, której celem jest zaspokajanie wspólnych i osobistych potrzeb. Jest to zespół osób, które wspólnie podejmują decyzję o swoim podstawowym utrzymaniu i zarządzaniu środkami pieniężnymi wniesionymi do budżetu domowego. Gospodarstwem domowym można także nazwać pojedynczą osobę utrzymującą się samodzielnie.

W dobie rosnącej konsumpcji szczególnie kluczowe jest zarządzanie finansami osobistymi i rodzinnymi. Ciągle kreowane przez producentów nowe potrzeby w łatwy sposób mogą sprawić, że konsument przestanie panować nad własnymi pieniędzmi. Łatwo może przestać odróżniać rzeczywiste potrzeby od zachcianek i stracić kontrolę nad budżetem. W konsekwencji takie nieodpowiedzialne zachowanie może doprowadzić do nadszarpnięcia, a nawet załamania całego budżetu domowego.

Tak samo jak panowanie nad wydatkami, istotne jest monitorowanie dochodów i inwestycji gospodarstwa domowego. Oprócz tradycyjnych dochodów z tytułu etatu, rodzina może mieć inne źródła przychodu jak np. praca dorywcza czy wynajem nieruchomości. Nieruchomości to niejedyna możliwość ulokowania oszczędności gospodarstwa domowego. Innymi przykładami inwestycji mogą być lokaty, obligacje skarbowe, złoto, a nawet akcje giełdowe.

Osoba mądrze kierująca gospodarstwem domowym szanuje pieniądze. Coraz częściej jednym ze sposobów ich zaoszczędzenia jest dołączanie do programów lojalnościowych, które wiążą się dodatkowymi rabatami oraz bonami na zakupy. Nierzadko też bogata oferta banków umożliwia odłożenie dodatkowej gotówki, czy zbieranie punktów w podobnych programach. W obu przypadkach warunkiem uczestnictwa w promocji jest wyrobienie dodatkowej karty, bądź pobranie i rejestracja w aplikacji mobilnej banku lub sprzedawcy.

Kolejnym ważnym aspektem domowych finansów jest trwałość zakupionych towarów takich jak np. ubrania oraz sprzęt RTV i AGD. Rozklejające się po kilku miesiącach buty, psująca się lodówka, jak również innego rodzaju uszkodzenia i usterki, mogą prowadzić do nadwyrężenia budżetu domowego. Dlatego w wiele domach trzyma się paragony i faktury w celu dokonania potencjalnej reklamacji. Spotykane są rozwiązania umożliwiające zarówno powiązania zakupów z kontem w aplikacji lojalnościowej, jak i dedykowane programy do trzymania dokumentów sprzedażowych w formie elektronicznej.

Jak widać dbanie o budżet gospodarstwa domowego nie jest rzeczą trywialną. Wielowymiarowość tego zadania może przyprawić osobę odpowiedzialną o zawrót głowy. Nie wystarczy tylko panować nad przychodami i wydatkami. Trzeba wziąć pod uwagę czynniki, takie jak programy lojalnościowe, które oferują atrakcyjne promocje, a także zabezpieczenie się przed utratą środków z powodu przedwczesnego zużycia się produktów. 

(Na rynku istnieje wiele rozwiązań ułatwiających te codzienne zadania.)

Istnieje wiele programów do zarządzania budżetem domowym. Jednym z przykładów jest: "Wallet - przychody i wydatki, karty lojalnościowe". Niestety jest ona ograniczona. Użytkownik w darmowej wersji może zdefiniować tylko trzy konta. Nie oferuje ona także możliwości integrowania wydatków innych członków rodziny, co uniemożliwia sprawny nadzór nad całym domowym budżetem.

W ostatnim czasie supermarkety czy sklepy odzieżowe wypuszczają własne oprogramowanie umożliwiające branie udziału w programach lojalnościowych. Jednym z nich jest Lidl, którego rozwiązanie oferuje też zapisanie paragonu w aplikacji. Cały proces odbywa się poprzez zeskanowanie kodu QR w telefonie. W tym wypadku problem staje się mnogość dedykowanych aplikacji dla każdego sklepu.

Popularną aplikacją, która pozwala na elektroniczne przechowywanie paragonów jest "Pan Paragon". Umożliwia on nie tylko na trzymanie dokumentów sprzedażowych w pamięci lokalnej telefonu, ale także na ich archiwizowanie w chmurze. Ułatwia on także na przechowywanie kart lojalnościowych. Niestety nie daje możliwości zarządzania wydatkami i planowania budżetu. Jest też aplikacją dedykowaną jednostce, a nie rodzinie.

\section{Cel i zakres pracy}
https://pl.wikipedia.org/wiki/Gospodarstwo_domowe
\section{Układ pracy}