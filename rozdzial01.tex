\chapter{Wstęp}
\label{chap:wstep}
\section{Wprowadzenie}
\label{sec:wprowadzenie}
Zarządzanie finansami to jeden z ważniejszych aspektów funkcjonowania każdego przedsiębiorstwa, bez względu na rodzaj prowadzonej działalności, a także liczbę pracowników. Rozliczanie przychodów i wydatków jest nie tylko bardzo często obowiązkiem, ale również pomocą w panowaniu nad przepływem środków finansowych. 

Jako szczególny typ przedsiębiorstwa można uznać gospodarstwo domowe, czyli po prostu rodzinę. Z ekonomicznego punktu widzenia jej celem jest zaspokajanie wspólnych i osobistych potrzeb. Rodzina jest też zespołem osób, które wspólnie podejmują decyzję o swoim podstawowym utrzymaniu i zarządzaniu wniesionymi do budżetu domowego środkami. Gospodarstwem domowym można także nazwać pojedynczą osobę utrzymującą się samodzielnie~\cite{gospodarstwo-domowe}.

W dobie rosnącej konsumpcji oraz na skutek celowego stosowania różnych technik pobudzania zapotrzebowania przez producentów panowanie nad wydatkami zaczyna sprawiać sporo trudności. Problemem staje się odróżnienie zachcianek od tego, co z punktu widzenia funkcjonowania rodziny jest niezbędne. W konsekwencji dość łatwo jest doprowadzić do załamania całego domowego budżetu.

Z panowaniem nad wydatkami bezpośrednio wiąże się monitorowanie dochodów i~inwestycji gospodarstwa domowego. Oprócz tradycyjnych dochodów z tytułu etatu, rodzina może mieć inne źródła przychodu, jak na przykład praca dorywcza czy wynajem nieruchomości. Nieruchomości to niejedyna możliwość ulokowania oszczędności gospodarstwa domowego. Innymi przykładami inwestycji mogą być lokaty, obligacje skarbowe, złoto, a nawet akcje giełdowe. Coraz częściej jednym ze sposobów ich zaoszczędzenia jest dołączanie do programów lojalnościowych, które wiążą się dodatkowymi rabatami oraz bonami na zakupy. Nierzadko też bogata oferta banków pozwala na odłożenie dodatkowej gotówki czy zbieranie punktów w podobnych programach. W~obu przypadkach warunkiem uczestnictwa w promocji jest wyrobienie dodatkowej karty bądź pobranie i rejestracja w aplikacji mobilnej banku lub sprzedawcy.

Kolejnym ważnym aspektem domowych finansów jest trwałość zakupionych towarów, takich, jak na przykład ubrania oraz sprzęt RTV i AGD. Rozklejające się po kilku miesiącach buty, psująca się lodówka, jak również innego rodzaju uszkodzenia i usterki, mogą prowadzić do nadwyrężenia budżetu domowego. Dlatego w wielu domach trzyma się paragony i faktury w celu dokonania potencjalnej reklamacji. Spotykane są rozwiązania umożliwiające zarówno powiązania zakupów z kontem w aplikacji lojalnościowej, jak i dedykowane programy do trzymania dokumentów sprzedażowych w formie elektronicznej.

% TO DO: proszę dołożyć cytowania (do stron domowych wymienonych niżej aplikacji).
Popularną aplikacją, która pozwala na elektroniczne przechowywanie paragonów jest ,,Pan Paragon''. Umożliwia on nie tylko na trzymanie dokumentów sprzedażowych w pamięci lokalnej telefonu, ale również na ich archiwizację w chmurze. Pozwala on także na przechowywanie kart lojalnościowych. Niestety, nie daje możliwości do zarządzania wydatkami i planowania budżetu. Jest też aplikacją dedykowaną jednostce, a nie rodzinie.

Istnieje też wiele programów do zarządzania budżetem domowym. Jednym z przykładów jest ,,Wallet - przychody i wydatki, karty lojalnościowe''. Niestety, zakres oferowanych w nim funkcji jest ograniczony. Użytkownik w darmowej wersji może zdefiniować tylko trzy konta. Nie ma też możliwości integrowania wydatków innych członków rodziny, co komplikuje sprawny nadzór nad całym domowym budżetem.

W ostatnim czasie również supermarkety czy sklepy odzieżowe wypuszczają własne oprogramowanie, umożliwiające branie udziału w programach lojalnościowych. Jednym z nich jest Lidl, którego rozwiązanie oferuje też zapisanie paragonu w aplikacji. Cały proces odbywa się poprzez zeskanowanie kodu QR (ang.~\emph{quick response}) w telefonie. W przypadku tego typu programów, problemem staje się mnogość rozwiązań dla pojedynczych sieci i marek.

Jak widać dbanie o budżet gospodarstwa domowego nie jest rzeczą trywialną. Wielowymiarowość tego zadania może przyprawić osobę odpowiedzialną o zawrót głowy. Dużym utrudnieniem jest brak możliwości synchronizowania działań całej rodziny w ramach jednego systemu. Ponadto nie wystarczy tylko panować nad przychodami i wydatkami. Trzeba wziąć pod uwagę czynniki, takie jak zabezpieczenie się przed utratą środków z powodu przedwczesnego zużycia się produktów, a także  programy lojalnościowe, które oferują atrakcyjne promocje. Tutaj sprawę komplikuje ilość potencjalnych kart i możliwość współdzielenia ich między członkami rodziny. 

W tej pracy dyplomowej zdecydowano się przedstawić projekt aplikacji budżetu domowego, która łączyć będzie wszystkie wyżej wymienione funkcje. Aplikacja ma stać się uniwersalnym narzędziem do zarządzania finansami całej rodziny, ze szczególnym uwzględnieniem roli rodziców i dzieci. Ważnymi funkcjami projektu mają być również te związane z tworzeniem kopii zapasowej paragonów, a także zarządzaniem gwarancjami i mnogością kart lojalnościowych.

\section{Cel i zakres pracy}
\label{sec:cel-zakres}
\subsection{Cel}
\label{subsec:cel}
Celem pracy jest stworzenie aplikacji do zarządzania budżetem domowym, oferującej użytkownikowi bezpieczne i płynne korzystanie z dostarczanych w niej funkcji.
Aplikacja ta w założeniach ma być produktem MVP (ang.~\emph{minimum valuable product}), czyli rozwiązaniem zapewniającym minimalną wartość użytkową.  

Z realizacją celu wiąże się opracowanie koncepcji uniwersalnej aplikacji dla domowego budżetu, która łączyłaby w sobie funkcje innych programów tego typu z~funkcjami programów przechowujących kopie paragonów, a także oferujących możliwość przechowywania kart lojalnościowych, bonów i gwarancji. 
Ponadto w pracy mają powstać założenia projektu elastycznego systemu o nowoczesnej architekturze, umożliwiającej łatwe dodawanie nowych funkcji i integrację z zewnętrznymi dostawcami usług. 

\subsection{Zakres}
\label{subsec:zakres}
W ramach pracy zbudowane zostaną dwie aplikacje webowe. Pierwsza z nich ma oferować graficzny interfejs użytkownika (ang.~\emph{graphical user interface}, GUI) wyświetlany w oknie przeglądarki internetowej, druga zaś ma dostarczyć interfejs programistyczny aplikacji (ang.~\emph{application programming interface}, API)

Zaimplementowane funkcje powinny umożliwiać tworzenie i zarządzania  rachunkiem, tudzież kontem pieniężnym, wraz z możliwością tworzenia wydatków i przychodów. Dodatkowo obsługiwane mają być: tworzenie i przypisywanie kategorii operacji pieniężnych, jak również bezpieczny i nowoczesny sposób rejestracji oraz logowania się użytkownika w systemie, zaimplementowany z wykorzystaniem mechanizmu tokenów.

Aplikacja zostanie dostarczona na środowisko produkcyjne stworzone na platformie Azure. Zostanie ona przetestowana w środowisku deweloperskim oraz produkcyjnym. Ponadto zostanie spisana dokumentacja API.

\section{Układ pracy}
\label{sec:uklad-pracy}
W rozdziale~\ref{chap:wstep} przedstawiono współczesne problemy związane z prowadzeniem domowych finansów i uzasadnienie konieczności stworzenia aplikacji budżetu domowego. Omówiono w~nim również cel pracy inżynierskiej oraz zakres wykonanych prac.

W rozdziale \ref{chap:zalozenia-projektowe} przedstawiono założenia projektowe wraz z~zarysem architektury systemu i~poszczególnych jego komponentów. Zawarto w nim także wymagania funkcjonalne i niefunkcjonalne aplikacji.

Rozdział~\ref{chap:know-how} poświęcono na wstęp teoretyczny. Opisano w nim bazę wiedzy zgromadzoną przez autora podczas przygotowań do realizacji pracy. W podrozdziale~\ref{sec:projektowanie-api} skupiono uwagę na~dobrych praktykach tworzenia API~oraz przydatnych narzędziach. W podrozdziale~\ref{sec:wzorce} przybliżono wykorzystane w~kodzie wzorce projektowe. W podrozdziale~\ref{sec:autoryzacja} dokonano prezentacji użytego mechanizmu autoryzacji i uwierzytelniania.

Szczegóły implementacji opisano w~rozdziale~\ref{chap:implementacja}. W podrozdziale~\ref{sec:struktura-projektu} przedstawiono fizyczną i logiczną strukturę projektu, a~w podrozdziale~\ref{sec:szczegoly-implementacji} -- detale realizacji aplikacji klienckiej (ang.~\emph{frontend}) oraz API działającego po stronie serwera (ang.~\emph{backend}).

Na omówienie testów API i aplikacji klienckiej poświęcono rozdział~\ref{chap:testy}. 

W ostatnim rozdziale~\ref{chap:podsumowanie} zamieszczono podsumowanie pracy wraz z uwagami odnośnie zdobytych doświadczeń.