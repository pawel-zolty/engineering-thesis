\chapter{Wstęp}
\section{Wprowadzenie}
Bez względu na rodzaj działalności, a także liczbę pracowników, zarządzanie finansami to jeden z ważniejszych aspektów funkcjonowania każdego przedsiębiorstwa. Rozliczanie przychodów i wydatków jest nie tylko bardzo często obowiązkiem, ale także pomocą w panowaniu nad przepływem pieniędzy wynikających z prowadzenia działalności gospodarczej. Jako szczególny typ przedsiębiorstwa można podać gospodarstwo domowe, czyli po prostu rodzinę.

Z ekonomicznego punktu widzenia, gospodarstwo domowe to jednostka gospodarująca, której celem jest zaspokajanie wspólnych i osobistych potrzeb. Jest to zespół osób, które wspólnie podejmują decyzję o swoim podstawowym utrzymaniu i zarządzaniu środkami pieniężnymi wniesionymi do budżetu domowego. Gospodarstwem domowym można także nazwać pojedynczą osobę utrzymującą się samodzielnie.

W dobie rosnącej konsumpcji szczególnie kluczowe jest zarządzanie finansami osobistymi i rodzinnymi. Ciągle kreowane przez producentów nowe potrzeby w łatwy sposób mogą sprawić, że konsument przestanie panować nad własnymi pieniędzmi. Łatwo może przestać odróżniać rzeczywiste potrzeby od zachcianek i stracić kontrolę nad budżetem. W konsekwencji takie nieodpowiedzialne zachowanie może doprowadzić do nadszarpnięcia, a nawet załamania całego budżetu domowego.

Tak samo jak panowanie nad wydatkami, istotne jest monitorowanie dochodów i inwestycji gospodarstwa domowego. Oprócz tradycyjnych dochodów z tytułu etatu, rodzina może mieć inne źródła przychodu jak np. praca dorywcza czy wynajem nieruchomości. Nieruchomości to niejedyna możliwość ulokowania oszczędności gospodarstwa domowego. Innymi przykładami inwestycji mogą być lokaty, obligacje skarbowe, złoto, a nawet akcje giełdowe.

Osoba mądrze kierująca gospodarstwem domowym szanuje pieniądze. Coraz częściej jednym ze sposobów ich zaoszczędzenia jest dołączanie do programów lojalnościowych, które wiążą się dodatkowymi rabatami oraz bonami na zakupy. Nierzadko też bogata oferta banków umożliwia odłożenie dodatkowej gotówki, czy zbieranie punktów w podobnych programach. W obu przypadkach warunkiem uczestnictwa w promocji jest wyrobienie dodatkowej karty, bądź pobranie i rejestracja w aplikacji mobilnej banku lub sprzedawcy.

Kolejnym ważnym aspektem domowych finansów jest trwałość zakupionych towarów takich jak np. ubrania oraz sprzęt RTV i AGD. Rozklejające się po kilku miesiącach buty, psująca się lodówka jak również innego rodzaju uszkodzenia i usterki, mogą prowadzić do nadwyrężenia budżetu domowego. Dlatego w wiele domach trzyma się paragony i faktury w celu dokonania potencjalnej reklamacji. Spotykane są rozwiązania umożliwiające zarówno powiązania zakupów z kontem w aplikacji lojalnościowej, jak i dedykowanych programy do trzymanie dokumentów sprzedażowych w formie elektronicznej.



\section{Cel i zakres pracy}
Czym jest wsparcie jakie mozna uzyskac od aplikacji
przyklady zastosowan cytowania;
potencjalne problemy
potrzeba bedaca zrodlem pomyslu
https://pl.wikipedia.org/wiki/Gospodarstwo_domowe
\section{Układ pracy}